% generated by GAPDoc2LaTeX from XML source (Frank Luebeck)
\documentclass[a4paper,11pt]{report}

\usepackage[top=37mm,bottom=37mm,left=27mm,right=27mm]{geometry}
\sloppy
\pagestyle{myheadings}
\usepackage{amssymb}
\usepackage[utf8]{inputenc}
\usepackage{makeidx}
\makeindex
\usepackage{color}
\definecolor{FireBrick}{rgb}{0.5812,0.0074,0.0083}
\definecolor{RoyalBlue}{rgb}{0.0236,0.0894,0.6179}
\definecolor{RoyalGreen}{rgb}{0.0236,0.6179,0.0894}
\definecolor{RoyalRed}{rgb}{0.6179,0.0236,0.0894}
\definecolor{LightBlue}{rgb}{0.8544,0.9511,1.0000}
\definecolor{Black}{rgb}{0.0,0.0,0.0}

\definecolor{linkColor}{rgb}{0.0,0.0,0.554}
\definecolor{citeColor}{rgb}{0.0,0.0,0.554}
\definecolor{fileColor}{rgb}{0.0,0.0,0.554}
\definecolor{urlColor}{rgb}{0.0,0.0,0.554}
\definecolor{promptColor}{rgb}{0.0,0.0,0.589}
\definecolor{brkpromptColor}{rgb}{0.589,0.0,0.0}
\definecolor{gapinputColor}{rgb}{0.589,0.0,0.0}
\definecolor{gapoutputColor}{rgb}{0.0,0.0,0.0}

%%  for a long time these were red and blue by default,
%%  now black, but keep variables to overwrite
\definecolor{FuncColor}{rgb}{0.0,0.0,0.0}
%% strange name because of pdflatex bug:
\definecolor{Chapter }{rgb}{0.0,0.0,0.0}
\definecolor{DarkOlive}{rgb}{0.1047,0.2412,0.0064}


\usepackage{fancyvrb}

\usepackage{mathptmx,helvet}
\usepackage[T1]{fontenc}
\usepackage{textcomp}


\usepackage[
            pdftex=true,
            bookmarks=true,        
            a4paper=true,
            pdftitle={Written with GAPDoc},
            pdfcreator={LaTeX with hyperref package / GAPDoc},
            colorlinks=true,
            backref=page,
            breaklinks=true,
            linkcolor=linkColor,
            citecolor=citeColor,
            filecolor=fileColor,
            urlcolor=urlColor,
            pdfpagemode={UseNone}, 
           ]{hyperref}

\newcommand{\maintitlesize}{\fontsize{50}{55}\selectfont}

% write page numbers to a .pnr log file for online help
\newwrite\pagenrlog
\immediate\openout\pagenrlog =\jobname.pnr
\immediate\write\pagenrlog{PAGENRS := [}
\newcommand{\logpage}[1]{\protect\write\pagenrlog{#1, \thepage,}}
%% were never documented, give conflicts with some additional packages

\newcommand{\GAP}{\textsf{GAP}}

%% nicer description environments, allows long labels
\usepackage{enumitem}
\setdescription{style=nextline}

%% depth of toc
\setcounter{tocdepth}{1}





%% command for ColorPrompt style examples
\newcommand{\gapprompt}[1]{\color{promptColor}{\bfseries #1}}
\newcommand{\gapbrkprompt}[1]{\color{brkpromptColor}{\bfseries #1}}
\newcommand{\gapinput}[1]{\color{gapinputColor}{#1}}


\begin{document}

\logpage{[ 0, 0, 0 ]}
\begin{titlepage}
\mbox{}\vfill

\begin{center}{\maintitlesize \textbf{ ModIsom \mbox{}}}\\
\vfill

\hypersetup{pdftitle= ModIsom }
\markright{\scriptsize \mbox{}\hfill  ModIsom  \hfill\mbox{}}
{\Huge \textbf{ Computing automorphisms and checking isomorphisms for modular group algebras
of finite p-groups \mbox{}}}\\
\vfill

{\Huge  3.0.0 \mbox{}}\\[1cm]
{ 26 February 2024 \mbox{}}\\[1cm]
\mbox{}\\[2cm]
{\Large \textbf{ Bettina Eick\\
    \mbox{}}}\\
{\Large \textbf{ Olexandr Konovalov\\
    \mbox{}}}\\
\hypersetup{pdfauthor= Bettina Eick\\
    ;  Olexandr Konovalov\\
    }
\end{center}\vfill

\mbox{}\\
{\mbox{}\\
\small \noindent \textbf{ Bettina Eick\\
    }  Email: \href{mailto://beick@tu-bs.de} {\texttt{beick@tu-bs.de}}\\
  Homepage: \href{http://www.iaa.tu-bs.de/beick} {\texttt{http://www.iaa.tu-bs.de/beick}}\\
  Address: \begin{minipage}[t]{8cm}\noindent
 Institut Analysis und Algebra\\
 TU Braunschweig\\
 Universit{\"a}tsplatz 2\\
 D-38106 Braunschweig\\
 Germany\\
 \end{minipage}
}\\
{\mbox{}\\
\small \noindent \textbf{ Olexandr Konovalov\\
    }  Email: \href{mailto://obk1@st-andrews.ac.uk} {\texttt{obk1@st-andrews.ac.uk}}\\
  Homepage: \href{https://alex-konovalov.github.io/} {\texttt{https://alex-konovalov.github.io/}}\\
  Address: \begin{minipage}[t]{8cm}\noindent
 School of Computer Science\\
 University of St Andrews\\
 Jack Cole Building, North Haugh,\\
 St Andrews, Fife, KY16 9SX, Scotland\\
 \end{minipage}
}\\
\end{titlepage}

\newpage\setcounter{page}{2}
\newpage

\def\contentsname{Contents\logpage{[ 0, 0, 1 ]}}

\tableofcontents
\newpage

     
\chapter{\textcolor{Chapter }{Introduction}}\label{Chapter_Introduction}
\logpage{[ 1, 0, 0 ]}
\hyperdef{L}{X7DFB63A97E67C0A1}{}
{
  

 This package contains various algorithms related to finite dimensional
nilpotent associative algebras. It also contains many group-theoretical
functions related to the Modular Isomorphism Problem. We first give a brief
introduction to finite dimensional nilpotent algebras and then an overview of
the main algorithms. 

 
\section{\textcolor{Chapter }{Associative algebras and nilpotency}}\label{Chapter_Introduction_Section_Associative_algebras_and_nilpotency}
\logpage{[ 1, 1, 0 ]}
\hyperdef{L}{X7DF90DDB804F3313}{}
{
  

 Let $A$ be an associative algebra of dimension $d$ over a field $F$. Let $\{b_1, \ldots, b_d\}$ be a basis for $A$. We identify the element $x_1 b_1 + \ldots + x_d b_d$ of $A$ with the element $(x_1, \ldots, x_d)$ of $F^d$. The multiplication of $A$ can then be described by a \emph{structure constants table}: a 3-dimensional array with entries $a_{i,j,k} \in F$ satisfying that 
\[b_i b_j = \sum_{k=1}^d a_{i,j,k} b_k.\]
 

 An associative algebra $A$ is \emph{nilpotent} if its \emph{power series} terminates at the trivial ideal of $A$; that is 
\[A > A^2 > \ldots > A^c > A^{c+1} = \{0\} \]
 where $A^j$ is the ideal of $A$ generated by all products of length at least $j$. The length $c$ of the power series is also called the \emph{class} of $A$ and the dimension of $A/A^2$ is the \emph{rank} of $A$. Note that $A$ is generated by $dim(A/A^2)$ elements. Clearly, $A$ does not contain a multiplicative identity. 

 For computational purposes we describe a nilpotent associative algebra by a
weighted basis and a description of the corresponding structure constants
table. A basis of a nilpotent associative algebra $A$ is \emph{weighted} if there is a sequence of weights $(w_1, \ldots, w_d)$ so that 
\[A^j = \langle b_i \mid w_i \geq j \rangle.\]
 Note that $A A^j = A^{j+1}$ for every $j$. Thus it is possible to choose all basis elements of weight at least 2 so
that $b_i = b_k b_l$ holds for some $k$ and $l$, where $b_k$ is of weight 1 and $b_l$ is of weight $w_i-1$. This feature allows an effective description of $A$ via a \emph{nilpotent structure constants table}. This contains the structure constants $a_{i,j,k}$ for all $i$ with $w_i = 1$ and $1 \leq j,k \leq d$. For $i$ with $w_i > 1$ it either contains a description as $b_i = b_k b_l$ or the structure constants $a_{i,j,k}$ for $1 \leq j,k \leq d$. It may also contain both or some partial overlap of these informations. 

 }

 
\section{\textcolor{Chapter }{Isomorphisms and Automorphisms}}\label{Chapter_Introduction_Section_Isomorphisms_and_Automorphisms}
\logpage{[ 1, 2, 0 ]}
\hyperdef{L}{X81F3496578EAA74E}{}
{
  

 Let $A$ be a finite dimensional nilpotent associative algebra over a finite field.
This package contains an implementation of the methods in \cite{Eick08} which allow the determination of the automorphism group $Aut(A)$ and a \emph{canonical form} $Can(A)$. 

 The automorphism group is given by generators and is represented as a subgroup
of $GL(dim(A), F)$. Also the order of $Aut(A)$ is available. 

 A canonical form $Can(A)$ for $A$ is a nilpotent structure constants table for $A$ which is unique for the isomorphism type of $A$; that is, two algebras $A$ and $B$ are isomorphic if and only if $Can(A) = Can(B)$ holds. Hence the canonical form can be used to solve the isomorphism problem. 

 }

 
\section{\textcolor{Chapter }{The Modular Isomorphism Problem}}\label{Chapter_Introduction_Section_The_Modular_Isomorphism_Problem}
\logpage{[ 1, 3, 0 ]}
\hyperdef{L}{X84DE9285877BEDEE}{}
{
  

 The modular isomorphism problem asks whether $\mathbb{F}_p G \cong \mathbb{F}_p H$ implies that $G \cong H$ for two $p$-groups $G$ and $H$ and $\mathbb{F}$ the field with $p$ elements. This problem was open for a long time until first counterexamples
for the prime $p=2$ were found in \cite{GarciaLucasMargolisDelRio22}. It remains open for odd primes and many other interesting classes of groups. 

 Computational approaches have been used to investigate the modular isomorphism
problem. Based on an algorithm by Roggenkamp and Scott \cite{RoggenkampScott93}, Wursthorn \cite{Wursthorn93} described an algorithm for checking the modular isomorphism problem; that is,
he described an algorithm for checking whether two modular group algebras $\mathbb{F} G$ and $\mathbb{F} H$ are isomorphic, where $G$ and $H$ are finite $p$-groups. This algorithm has been implemented in C by Wursthorn and has been
applied to the groups of order dividing $2^7$ without finding a counterexample, see \cite{BleherKimmerleRoggenkampWursthorn99}. The implementation of Wursthorn appears lost, but is in any case not
publicly available. 

 This package contains an implementation of the new algorithm described in \cite{Eick08} for checking isomorphism of modular group algebras. It is based on the fact
that the Jacobson radical $J(FG)$ is nilpotent if $FG$ is a modular group algebra for $G$ a finite $p$-group and $FG$ is isomorphic to $FH$ if and only if the radicals $J(FG)$ and $J(FH)$ are isomorphic. Hence the automorphism group and canonical form algorithm of
this package apply and can be used to solve the isomorphism problem for
modular group algebras of finite $p$-groups. 

 The methods of this package have been used to study the modular isomorphism
problem for the groups of order dividing $3^6$ and $2^8$ (\cite{Eick08}) and for the groups of order $2^9$ (\cite{EickKonovalov11}). It was later used to study also groups of order $3^7$ and $5^6$ (\cite{MargolisMoede22}). 

 A property of a group $G$ is called \emph{$F$-invariant}, if an isomorphism of $F$-algebras $FG \cong FH$ implies the same property for $H$. In the context of the Modular Isomorphism Problem, if $G$ is a finite $p$-group, then an $\mathbb{F}_p$-invariant is simply called \emph{invariant}. Many invariants of $G$ are known and the package provides functions for them, as well as programs
which easily allow to compare all the implemented invariants quickly for a
given list of groups. 

 It also remains open, if replacing the field $\mathbb{F}_p$ in the Modular Isomorphism Problem with a bigger field of characteristic $p$ will change the outcome of the problem for a given pair of groups. The package
includes several functions which allow to investigate this question by
applying the algorithm for the same groups varying the field. 

 }

 
\section{\textcolor{Chapter }{A nilpotent quotient algorithm}}\label{Chapter_Introduction_Section_A_nilpotent_quotient_algorithm}
\logpage{[ 1, 4, 0 ]}
\hyperdef{L}{X7C156D9A7E2DA265}{}
{
  

 Given a finitely presented associative algebra $A$ over an arbitrary field $F$, this package contains an algorithm to determine a nilpotent structure
constants table for the class-$c$ nilpotent quotient of $A$, i.e. the algebra $A/A^{c+1}$. See \cite{Eick11} for details on the underlying algorithm. 

 }

 
\section{\textcolor{Chapter }{Kurosh Algebras}}\label{Chapter_Introduction_Section_Kurosh_Algebras}
\logpage{[ 1, 5, 0 ]}
\hyperdef{L}{X803522C37B413305}{}
{
  

 Let $F(d,F)$ denote the free non-unital associative algebra on $d$ generators over the field $F$. Then 
\[A(d,n,F) = F(d,F) / \langle \langle w^n \mid w \in F(d,F) \rangle \rangle\]
 is the \emph{Kurosh Algebra} on $d$ generators of exponent $n$ over the field $F$. Kurosh Algebras can be considered as an algebra-theoretic analogue to
Burnside groups. 

 This package contains a method that allows to determine $A(d,n,F)$ for given $d$, $n$, $F$. This can also be used to determine $A(d,n,F)$ for all fields of a given characteristic. We refer to \cite{Eick11} for details on the algorithms. 

 This package also contains a database of Kurosh Algebras that have been
determined with the methods of this package. 

 }

 
\section{\textcolor{Chapter }{Test section}}\label{Chapter_Introduction_Section_Test_section}
\logpage{[ 1, 6, 0 ]}
\hyperdef{L}{X7AEF443A7E504972}{}
{
  

 Here I want to test commands 

 }

 }

 \def\bibname{References\logpage{[ "Bib", 0, 0 ]}
\hyperdef{L}{X7A6F98FD85F02BFE}{}
}

\bibliographystyle{alpha}
\bibliography{ModIsom.bib}

\addcontentsline{toc}{chapter}{References}

\def\indexname{Index\logpage{[ "Ind", 0, 0 ]}
\hyperdef{L}{X83A0356F839C696F}{}
}

\cleardoublepage
\phantomsection
\addcontentsline{toc}{chapter}{Index}


\printindex

\immediate\write\pagenrlog{["Ind", 0, 0], \arabic{page},}
\immediate\write\pagenrlog{["Ind", 0, 0], \arabic{page},}
\immediate\write\pagenrlog{["Ind", 0, 0], \arabic{page},}
\immediate\write\pagenrlog{["Ind", 0, 0], \arabic{page},}
\newpage
\immediate\write\pagenrlog{["End"], \arabic{page}];}
\immediate\closeout\pagenrlog
\end{document}
